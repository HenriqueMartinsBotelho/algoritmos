\documentclass{exam}

\usepackage[12pt]{extsizes}
\usepackage[a4paper,left=2cm,right=2cm,top=2cm,bottom=2cm]{geometry}
\usepackage[utf8]{inputenc}
\usepackage[T1]{fontenc}
\usepackage{lmodern} 
\usepackage{amsthm,amsmath, amssymb}
\newcommand{\R}{\ensuremath{\mathbb{R}}}
\usepackage{float}
\usepackage{graphicx}
\usepackage{caption} %para usar o caption sem numeracao automatica
\usepackage{booktabs}
\usepackage{tikz}
\usetikzlibrary{arrows,positioning,shapes,fit,calc}

\renewcommand{\solutiontitle}{\noindent\textbf{Solução: }\noindent}

\newtheorem{Teorema}{Teorema}
\newtheorem{Lema}{Lema}

\printanswers        %DESCOMENTAR PARA MOSTRAR AS RESPOSTAS
\begin{document}
	
	\section{"Solução" de alguns exercícios de algoritmos e estrutura de dados}
	
	\cancelspace
	
	\begin{questions}
		
		\question  É verdade que $4n^2 = O(n^2)$ 
		\begin{solution}[.2in]
			Sim. Temos que $4n^2 \le 4n^2$, então basta definir: $C = 4, n_0 = 1$.
		\end{solution} 
		
		\question É verdade que $2^{0.694} = O(2^n)$
		\begin{solution}
			Sim, pois $2^{0.694} \le 1.2^n \; \forall n > 1$. $C = 0, \;\ n_0 = 1$.
		\end{solution}
		
		\question É verdade que $10n^2 + 5n + 3 = O(n^2)$?
		\begin{solution}
			Sim, de fato temos: $10n^2 + 5n + 3 \le 10n^2 + 5n^2 + 3n^2 = (10 + 5 + 3).n^2 = 18n^2 = O(n^2)$, com $C = 18$ e $n_0 = 1$.
		\end{solution}
	
		\question É verdade que $7n^2 = O(n)$?
		\begin{solution}
			Não. Suponha que existam $c,n_0$ tal que $7n^2 \le c.n, \; \forall n > n_0$. Então dividindo ambos os lados da desigualdade
			por $n$ temos $7n \le c, \; \forall n > n_0$. O que é absurdo pois $c$ é constante e $7n$ tende a infinito.
		\end{solution}
		
		\question É verdade que $\frac{1}{2}n^2 + 3n = O(n)$?
		\begin{solution}
			Não.
		\end{solution}
	
		\question É verdade que $\frac{1}{2}n^2 + 3n = O(n^3)$?
		\begin{solution}
			Sim.
		\end{solution}

		\question É verdade que $n^k = O(n^{k-1})$?
		\begin{solution}
		Não. Suponha que existam $c,n_0$ tal que $n^k \le c . n^{k-1}$. Então temos $\frac{n^k}{n^{k-1}} \le c$ e 
		daí que $n \le c$, para todo $n > n_0$. Absurdo.  
		\end{solution}
	
		\question É verdade que $a_kn^k + \cdots + a_1n + a_0 = O(n^k)$?
		\begin{solution}
			Sim, pois  $a_kn^k + \cdots + a_1n + a_0 \le a_kn^k + \cdots + a_1n^k + n^k =
			(a_n + a_{n-1} \cdots a_0)n^k = O(n^k)$.
		\end{solution}
				
		\question É verdade que $100n + \log n = \Theta(n + \log^2 n)$?
		\begin{solution}
			Sim, $100n + \log n \le 100n + 100 \log^2 n = 100(n + log^2 n)$, temos
			$c = 100$ e $n_0 = 1$. 
		\end{solution}
	
		\question $\log 2n = \Theta(\log 3n)$?
		\begin{solution}
			Provemos primeiro que é $O(\log 3n)$:\\
			$\log 2n = \log 2 + \log n$ e $\log 3n = \log 3 + \log n$. Como $\log 3 > \log 2$ podemos concluir que
			$\log 2n \le \log 3n$ que implica em $\log 2n \le 1. \log 3n \;\; \forall n > 1$. Logo, $c = 1$ e $n_0 = 1$.\\
			Agora, veremos que também é $\Omega$. Temos $\log 2n < \log 3n$ queremos achar $c$ tal que $\log 2n \ge c . \log 3n$ 
			seja verdade, então se tomarmos $c = \frac{\log 2n}{\log 3n}$ ou menor temos: $\log 2 \ge \log 3n. \frac{\log 2n}{\log 3n}
			\; \forall n > 1$.
		\end{solution}
	
		\question É verdade que $\log_a n = \Theta(\log_b n)$ com $a$ e $b$ inteiros positivos?
		\begin{solution}
			Sim, é verdade. Sem perca de generalidade suponhamos $\log_a n < \log_b n $ e daí:
			Primeiro queremos achar $c$ tal que $$\log_a n \le \log_b n$$ para isso tomemos
			$c = \frac{log_a n }{log_b n}$ ou menos e temos garantidamente:
			$$\log_a b \le \log_b n.\frac{\log_a n}{\log_b n}$$
			Analogamente, se $c = \frac{\log_b n}{\log_a n}$ ou maior temos: 
			$$\log_a n \ge log_b n. \frac{\log_b n}{\log_a n}$$ 
			
			
		\end{solution}

		\question É verdade que $(n + a)^b = \Theta(n^b)$ com $a$ e $b$ inteiros positivos?
		\begin{solution}
			acho que dá para fazer por indução em $b$.
		\end{solution}
	
		\question É verdade que $4 \log n + 3 \sqrt{n} + 5n^2 = \Theta(n^2)$?
		\begin{solution}
			É verdade.
		\end{solution}
	
		\question É verdade que $2^{5n} = O(2^n)$?
		\begin{solution}
			Sim, $2^{5n} = (2^5)^n > c.2^n$, pois $c$ é uma constante. 
		\end{solution}







\end{questions}
		
\end{document}		
		
		
